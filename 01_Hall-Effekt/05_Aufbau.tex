% !TeX root = Bericht.tex
% !TeX spellcheck = de_DE

\section{Aufbau}

Zur Messung der Elementarladung bzw. der spezifischen Ladung wird eine Ladungsspektroskopie durchgeführt. Dies wird über das Zusammenspielen von Zentripetalkraft und Lorentzkraft erreicht. Der dafür benützte Aufbau ist ein Fadenstrahlrohr.  Dieses besteht aus einer Helmholtz-Spule in deren Inneren ein evakuierter Glaskolben liegt. Der Restdruck ist etwa $1 \unit{Pa}$ Wasserstoffgas. Der Druck wird klein gehalten, um einen möglichst ungestörten Elektronenfluss zu gewährleisten. 

Im Kolben ist auf der rechten Seite eine Metallspitze die nach oben zeigt und eine Metallscheibe die unterhalb positioniert ist angebracht. Über die Platte fällt ein Potential $U$ auf die Spitze ab. Letztere wird über eine kleine Spannung erhitzt, welches zu einem Austreten von Elektronen an der Spitze führt, die daraufhin durch das Potential beschleunigt werden. Um den Strahl zu fokussieren, sind zwei weitere Potentialplatten knapp über der Spitze angebracht. Diese sollen für einen Möglichst schmalen Elektronenstrahl sorgen. 

Hinter der zweiten Spule ist ein Spiegel befestigt und vor der ersten sind zwei vertikale Stäbe angebracht. 
Diese kann man horizontal auf einer Skala verschieben. Bei der Durchführung positioniert man den einen Stab auf der rechten Seite des zu sehenden Kreises so, dass dieser sein Spiegelbild und das Maximum verdeckt. Das heißt der Stab, das Maximum des Strahles und die Reflexion des Stabes sind in einer Sichtlinie. Der Stab auf der linken Seite wird analog positioniert und somit kann der Durchmesser über die Skala bestimmt werden (Siehe \autoref{fig:Aufbau_Maschine}).


\begin{figure}
	\includegraphics[width=\textwidth]{Aufbau_Maschine}
	\caption{Zeigt das Fadenstrahlrohr, im inneren Bildet sich ein Glühender Ring der mit den zwei vertikalen Stäben abgemessen wird.}
	\label{fig:Aufbau_Maschine}
\end{figure}


Die Beschleunigungsspannung $U$, Platte über Spitze, wird von $127 \unit{V}$ bis $233 \unit{V}$ variiert. Alle anderen Spannungen und Ströme wurden über den Versuch konstant gehalten. Also einen Spulenstrom von $1.07 \unit{A}$ und eine Heizspannung von $6.3 \unit{V}$.  Über zwei Multimeter wird sowohl Hauptspannung als auch Spulenstrom gemessen. 

Für alle eingestellten Werte, wird von den drei Teammitgliedern unabhängig voneinander eine Messung durchgeführt. Dies geschieht um systematische Fehler zu reduzieren. Die Spannungs- und Stromwerte werden nach einstellen abgelesen und notiert, bevor die drei Messungen der Versuchsteilnehmer durchgeführt werden. 
