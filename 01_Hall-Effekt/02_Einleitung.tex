% !TeX root = Bericht.tex
% !TeX spellcheck = de_DE
\section{Einleitung}
Halbleiterbauteile sind ein essentieller Bestandteil der heutigen Welt. Unmöglich wären moderne Computer und Mobiltelefone, sowie jede andere Art der Elektronik ohne den Einsatz von Halbleitern, ja selbst eine moderne Modelleisenbahn kommt ohne Sie nicht aus. Doch wo finden sich Halbleiter in elektrotechnischen Bauteilen? Prominente Vertreter sind verschiedenste Dioden, MOSFETs und Transistoren. Die zahlreichen Anwendungsfelder werfen die Frage nach der Besonderheit der Eigenschaften von Halbleitern auf. Einige Eigenschaften einer Halbleiter-Probe werden in diesem Versuch untersucht. \newline
In diesem Bericht wird nach einer kurzen Beschreibung der nötigen Theorie die Durchführung des Versuches erläutert. In der Datenauswertung werden mithilfe der „van der Pauw“-Methode Spannungen gemessen und mit Magnetfeldmessungen kombiniert .Der Hall-Koeffizient und der spezifische Widerstand einer Probe lassen sich daraus berechnen, um anschließend die Ionisationsenergie der Donatoren bzw. Akzeptoren zu bestimmen und den Ladungsträgerüberschuss in Abhängigkeit der inversen Temperatur aufzutragen. Aus diesem sowie dem temperaturabhängigen Verhalten lässt sich das Material der Probe bestimmen. Weiters wird noch ein anderer Aspekt in der Datenauswertung beleuchtet, und zwar wird der dominante Streumachanismus im Hoch- und Niedertemperaturbereich bestimmt. Zusammengefasst bieten bei diesem Versuch simple Messungen, nämlich Spannungsmessungen, einen spannenden Einblick in die Halbleiterphysik.
