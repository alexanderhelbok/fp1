% !TeX root = Bericht.tex
% !TeX spellcheck = de_DE
\section{Diskussion}
Die Analyse stellt uns vor ein Problem. Von den 5 Messreihen a 3 Datenpunkte liegen 4 davon alle innerhalb einer Standardabweichung, während Messreihe 4 bei allen Versuchsteilnehmern signifikant davon abweicht. Nachdem das bei allen Personen der Fall ist (und diese untereinander konsisten sind) schließen wir einen statistische Ursache aus. Eine Erklärung könnte sein, dass beim notieren des Spannungswertes (welches zu Beginn jeder Messreihe also nur einmal für alle drei Messungen stattfindet) uns ein menschlicher Fehler unterlaufen ist und eine Ziffer falsch notiert wurde. Beim Aufschreiben des Radius kann das aber nicht passiert sein, weil dann der selbe Abschreibfehler bei allen 3 Messungen geschehen sein muss (und das gleich!), was wir als unwahrscheinlich ansehen. Anzumerken ist noch, dass der inkonsistente Wert beim kleinsten Radius aufgenommen wurde, evtl. stecken doch Systematiken dahinter, die wir aber nicht ausfindig machen konnten. 

Aus obigen Gründen werden hier die Ergebnisse präsentiert, wo der inkonsistente Messpunkt nicht berücksichtigt wurde. Für die spezifische Ladung des Elektrons ergibt sich aus der Steigung des Fits \( e/m_{\text{e}} = 1.92(4) \cdot 10^{11} \unit{C/kg} \). Vergleicht man diesen Wert mit dem Literaturwert aus dem CODATA Index \( 1.758 820 010 76 (53) \cdot 10^{11} \unit{C/kg} \) \cite{codata}, erkennt man, dass wir ungefähr 4 Standardabweichungen davon entfernt sind. Die Abweichung kann statistisch noch begründet werden, ist aber eher unwahrscheinlich. Die bessere Erklärung sind etwaige Systematiken. Diese wurden analysiert und wir kommen zum Schluss, dass wir durch eine kleine Nichtparallelität des Spiegels (etwa \( \ang{3} \)) auf den Literaturwert kommen. 

Die Elementarladung haben wir auf \( e = 1.75(2) \cdot 10^{-19} \unit{C} \) bestimmt. Der Literaturwert \( 1.602 176 634 \unit{C} \) \cite{codata} liegt mehrere Standardabweichungen vom experimentell bestimmten Wert entfernt. Diese Diskrepanz liegt wahrscheinlich wieder etwaigen Systematiken zugrunde. 
