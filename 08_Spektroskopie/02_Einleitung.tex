% !TeX root = Bericht.tex
% !TeX spellcheck = en_US
\section{Introduction}
Ever since the dawn of human reasoning, the sun has been a fascinating object. From astrology to mythology it has found its way into everyone's lives. But apart from speculations and pseudo-science, the sun has led to many scientific advancements, with the first estimation of the earths diameter being one of many. Here, Eratosthenes, a Greek philosopher, used an ingenious technique, comparing the shadows thrown by an obelisk in Alexandria and a deep well shaft in Syrene to determine the earth's diameter with an impressive error of only $3 \%$ in $240$ BC. 

Even more insights were gained by delving deeper into our sun's light. Helium, the second most basic atom in the periodic table was first discovered in 1868 in the sun's spectrum. Anomalous absorption lines were observed that did not match any transition of known elements and so a new element had been discovered. 

With so much information encoded in sunlight and still many unanswered regarding our hot companion, we also want to take a crack at unraveling our sun's mysteries by recording, correcting and scaling a solar spectrum. This allows us to determine the effective temperature of the Sun and the iron-to-hydrogen ratio in the Sun's atmosphere.
