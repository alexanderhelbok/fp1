% !TeX root = Bericht.tex
% !TeX spellcheck = de_DE
\section{Interpretation}
We successfully recorded, corrected and calibrated a solar spectrum. We then proceeded to get a first estimate of the suns temperature $T = 6100(300) \unit{K}$ and iron-to-helium abundance $\NFe/\NH = 0.011(4) \unit{\percent}$. Both differ significantly from literature values ($T = 5772 \unit{K}$ and $\NFe/\NH = 0.00473 \unit{\percent}$ \autocite{NASA}) and a possible error source was identified. The first estimate used absorption lines, whose effective widths follow two different behaviours. 

By choosing a suitable representation of the data we successfully identified both regimes and excluded optically thick absorption lines. The refined data was again used to estimate some of the sun's parameters, yielding an effective temperature of $T = 5520(110) \unit{K}$ and $\NFe/\NH = 0.043(7) \unit{\percent}$. Whilst the reduced data follows the theoretical trend clearer, the extracted values did not really improve. This indicates the presence of error sources that dominate over the error introduced by using data from different proportionality regimes. A possible and easily verifiable cause could be the weather that may introduce nonlinearities by influencing the effective width of absorption lines differently, depending on the wavelength.
