% !TeX root = Bericht.tex
% !TeX spellcheck = en_US
\section{Introduction}

Without lasers, the world would look quite different from the world we know. From GPS to accurate clocks, lasers are omnipresent in the modern world, essential in a big variety of applications. 
But not only technical applications are worth mentioning here; they also play an indispensable
role in various physical experiments, shedding light on phenomena ranging from quantum mechanics to material science. In this experiment, we created setups, in which lasers are combined with an electro-optical-modulator (EOM). This component contains a crystal, which changes its optical properties when exposed to an electric field. This behaviour is called the electro-optical effect, which can be applied in many ways. For example, one can create lenses with adaptive focal length as well as optical switches. In the first setup, a Mach-Zehnder-interferometer is set up, with an EOM placed in one of its arms. The interference pattern of the output beam is recorded by a photodiode, enabling us to determine the half wave voltage and the bandwidth. Afterwards, an optical switch is realized in a simpler setup, where a laser beam passes an EOM and finally enters a photodiode. Before and after the EOM, waveplates and polarizers are used to inspect the behavior of different polarisation's of the laser light. 

The first chapter of this report deals with the necessary theory, followed by the experimental setup and procedure. In the next chapter, the results and data analysis are presented. Finally, the results are interpreted. 