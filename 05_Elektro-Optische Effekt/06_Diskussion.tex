% !TeX root = Bericht.tex
% !TeX spellcheck = en_US
\section{Interpretation}

The first series of measurements with the Mach-Zehnder interferometer, where no waveplate is used, supports theory. Our result for the coefficient $r_{13}$ of the pockels tensor is \SI{8.9(5)}{pm/V}, whereas \SI{8.6(2)}{pm/V} was given in the script. Thus, both values equal each other within uncertainty. 
When comparing the experimental value for the extraordinary axis, $r_{33}$, \SI{19.4(1.1)}{pm/V}, to the value in the script, \SI{30.8(2)}{pm/V}, a significant deviation is recognized. This could be a result of a imprecise placement of the $\lambda/2$-waveplate. Maybe it was not adjusted exactly enough to transform the beam completely from the ordinary to the extraordinary axis of the EOM, what could be a reason that $r_{33}$ is much too small. 

The measurement of contrast dependent on frequency resulted in a bandwith of \SI{425(10)}{Hz}. The contrast dropped approximatel exponential in the observed interval. 
Nevertheless, the procedure of the experiment could be improved by some changes. A more precise adjustment of all optical elements would lead to better results. If the contrast could be increased even more, the results were more accurate. As the setup of the interferometer is very sensitive, it has to be avoided to touch the table or even to speak, as this influences the result. 

The second setup, in contrast, is much more stable and is not influenced that easy. 
With regard to the modulation of the circularly polarised laser, it can be stated that the expected half-wave voltage, evaluated with \autoref{EQ:VPI}, is $V_{\pi,theo}=149.6(2) \unit{V}$, while the measured value is  $V_\pi=152,16(7) \unit{V}$. The discrepancy can be attributed to the aforementioned uncertainty in determining $V_\pi$ with the $\lambda/2$ waveplate. With regard to the realisation of the optical switch, it can be said that it works well. It should be noted that the on/off ratio depends significantly on $V_\mathrm{OFF}$. The smaller it is, the larger this value is. This ratio diverges towards 0 for the limit value.  The flank analysis was conducted with the objective of identifying the source of the majority of the delay. This was subsequently determined during the measurement process. It was found that the propagation velocity within the crystal plays a minor role.  The value for the medium reaction of the photodiode is given by $14 \unit{ns}$ see \autocite{photodiode}. This is also significantly less than half the lifetime of the entire system. Another limiting factor is the function generator, which requires a certain amount of time to realise the edge of a rectangular signal. By reducing the high-impedance input resistance of the oscilloscope through the use of a resistor in parallel, the reaction time of the structure was significantly decreased.

It can therefore be concluded that this is the primary cause of the delay. To sum up, the main area for improvement is in the adjustment of the waveplate.