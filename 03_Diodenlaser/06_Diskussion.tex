% !TeX root = Bericht.tex
% !TeX spellcheck = en_US
\section{Discussion}
The measured output power of the laser diode at three different temperatures resulted in a very similar efficiency $\eta$ and differential quantum efficiency $\eta_d$ derived from it. This is in perfect agreement with theory. As expected, the threshold current $I_\mathrm{th}$ is increasing with rising temperatures, due to bigger internal losses at higher temperatures. 

When expanding our experimental setup to include a Fabri-Perot interferometer (FPI), we can clearly see the different modes from the FPI, as well as the internal laser modes from the diode. Assuming the FPIs free spectral range to be $\text{FSR}_\text{FPI} = 630 \unit{GHz}$, we can convert the oscilloscopes time domain measurements into frequencies, enabling us to analyze some frequency characteristics of the diode. By analyzing the distance between adjacent intensity peaks, the internal mode spacing of the laser diode was determined to be 
$\text{FSR}_\text{diode} = 62(2) \unit{GHz}$. 

Furthermore we investigated the impact of temperature and laser current on the frequency. This knowledge allows for accurate tuning of the output frequency. We can clearly see an overall decrease in frequency when increasing both temperature and current, since both increase the optical length of the laser, decreasing the frequency. The exact behavior however is quite unpredictable, since we are hopping between modes due to a changing gain profile  of the laser.

Although both control parameter have the same qualitative effect on frequency, changing current can be done more  precisely whilst also having a smaller impact on frequency. Temperature adjustments should therefore be reserved for coarse laser tuning, if at all, whilst current bias is preferable for fine tuning. 