% !TeX root = Bericht.tex
% !TeX spellcheck = de_DE
\section{Introduction}
Let's explore some applications of an Acoustic-Optical Modulator (AOM). Why is this of importance? On the one hand, AOM's serve as swift switches for lasers, enabling rapid operation. On the other hand, this modulator is capable of generating pulsed laser beams and beams with reduced intensity. 
\\
\noindent This experiment leverages a fundamental optical effect to achieve these valuable features: Bragg's law. But unlike diffraction on a crystal structure/diffraction grating, where the diffraction properties are fixed to material constants, in an AOM one can tune them. By coupling an electric signal to the crystal (\ch{TeO2}) via a piezoelectric transducer one can induce density waves into the crystal, locally changing the refractive index. With the adjustment of the amplitude (peak-to-peak voltage), as well as the frequency of the electric signal, it is possible to tune the diffraction behavior to fit ones needs. A primary objective of the experiment is to ascertain the impact of uncertainties and whether they align with theoretically expected behaviours. These uncertainties may involve angular deviations. Additionally, we analyse the AOM's behaviour across various frequencies.
\\
\noindent Following a brief theoretical overview and an explanation of the applied theory, this report will present and discuss the evaluated data.
