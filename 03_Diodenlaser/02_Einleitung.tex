% !TeX root = Bericht.tex
% !TeX spellcheck = en_US
\section{Introduction}

What would a world without laser technology look like? Well, it would look really different than the world we know. From GPS to accurate clocks, lasers are omnipresent in the modern world. But not only technical applications are worth mentioning here; they also play an indispensable role in various physical experiments, shedding light on phenomena ranging from quantum mechanics to material science. In a recent experiment, we delved into the behaviors of a diode laser, a type known for its affordability and compact design.

In this analysis, our focus is twofold. Firstly, we examine the power output of the diode laser as a function of the current applied to it. Through this investigation, we also scrutinized the temperature dependency of the threshold current, providing insight into the intricate relationship between temperature and laser performance. Secondly, we explore the frequency dependency of the laser output. To accomplish this, we employ an optical cavity, which facilitates the measurement of specific frequency branches. This technique not only allows for precise frequency analysis but also enables spectroscopic studies, providing a deeper understanding of the laser's spectral characteristics and potential applications in various fields.
Following a brief theoretical overview and an explanation of the applied theory, this report will present and discuss the evaluated data.