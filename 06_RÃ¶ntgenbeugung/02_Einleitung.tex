% !TeX root = Bericht.tex
% !TeX spellcheck = de_DE
\section{Einleitung}
Röntgenstrahlung findet Anwendungen in zahlreichen unterschiedlichen Bereichen. Nicht nur in der medizinischen Bildgebung, auch in Materialwissenschaften oder auch Sicherheitskontrollen an Flughäfen wird Röntgenstrahlung angewandt.  In der Physik finden Effekte der Röntgenstrahl-ung konkreter Anwendung in der Spektroskopie in der Astrophysik. Diese zahlreichen Anwen-dungsmöglichkeiten motivieren, sich mit Röntgenstrahlen experimentell genauer auseinandersetzen. Im ersten Versuchsteil wird die Zählrohrcharakteristik eines Geiger-Müller-Zählers analysiert, indem die Spannung des Plateaubereichs ermittelt wird. Danach erfolgt eine Selektion der Wellenlängen mithilfe eines LiF-Kristalls, um ein Spektrum aufzunehmen und damit die charakteristische Wellenlänge zu ermitteln. im dritten Versuchsteil wird das Plancksche Wirkungsquantum bestimmt, indem ein Röntgenspektrum für verschiedene Anodenspannungen aufgenommen wird. Im letzten Teil des Versuchs folgt eine Analyse der Absorption von Röntgenstrahlen in unterschiedlichen Medien. Dabei werden die Massenabsorptionskoeffizienten von Aluminium, Zink und Zinn bestimmt, sowie die Absorptionskoeffizienten von Kupfer und Nickel. Anhand dieser letzten Messung kann auch die Energie der K-Schalen der Atome berechnet werden.  
Nach einer Erklärung der für die Auswertung nötigen Theorie folgt eine Beschreibung des Versuchsaufbaus und der Durchführung. Daraufhin werden die Resultate präsentiert und interpretiert.

