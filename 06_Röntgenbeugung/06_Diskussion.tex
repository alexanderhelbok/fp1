% !TeX root = Bericht.tex
% !TeX spellcheck = de_DE
\section{Schlussfolgerung}

Die Bestimmung der Zählrohrcharakteristik ergab eine Spannung von $U_{\mathrm{min}}$=\SI{322(4)}{V}. Darüber bleibt die Zählrate bis \SI{500}{V} annähernd konstant, was dem erwarteten Verhalten entspricht. 
Das Spektrum der Wolframanode ergibt drei klar erkennbare Peaks, diese befinden sich im Rahmen der Unsicherheit an der gleichen Stelle wie die Literaturwerte. Bei zwei kleinen Peaks kann zwar wegen der zu geringen Auflösung die Position nicht mithilfe eines Fits bestimmt werden, es sind jedoch alle Peaks wie in der Literatur vorhanden. 

Die Bestimmung des Planckschen Wirkungsquantums bestätigt die Messung, da der literaturwert innerhalb der Unsicherheit unseres Messergebnisses liegt. Ein noch präziseres Ergebnis könnte durch eine höhere Anzahl an Datenpunkten realisiert werden, da hier mit nur sieben Datenpunkten etwas wenig vorliegen. Vor allem könnte die Unsicherheit des Ergebnisses so minimiert werden. 

Bei den Messungen mit unterschiedlich dicken Absorberfolien zeigen sich in den bestimmten Massenabsorptionskoeffizienten gravierende Abweichungen von den Literaturwerten. in möglicher Grund wären Verunreinigen und kleine Beschädigungen der sehr sensitiven Folien. 

Aus den letzten Messungen mit Kupfer und Nickel wird zudem das Energieniveau der K-Schale des jeweiligen Elements bestimmt. Für Nickel stimmen Literatur- und Messwert im Rahmen der Unsicherheit überein, bei Kupfer zeigt sich eine sehr geringe Abweichung, die möglicherweise auch auf Verunreinigungen der Folie zurückzuführen ist. 