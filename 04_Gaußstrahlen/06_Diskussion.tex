% !TeX root = Bericht.tex
% !TeX spellcheck = en_US
\section{Discussion}
We successfully determined the HeNe laser's beam waist to $w_0 = 224(2) \unit{\micro\m}$, with data following the predicted behavior. When adding a lens, again the theoretically calculated value $w_{1, \mathrm{theo}} = 83.3(1.1) \unit{\micro\m}$ and the experimental $w_{1,\mathrm{exp}} = 82(3) \unit{\micro\m}$ are consistent with each other. Here, the datapoints deviated somewhat from the expected line, which can probably be traced back to inaccuracies in the measurement setup. Mounting the waistmeter on a movable rail would aid in keeping the waistmeter orthogonal to the incident laser beam, which is paramount for accurate readings. 

After simulating and successfully matching the laser beam to the cavity's internal modes, we validated the cavity to be confocal by measuring the distance between \blockquote{regular} and half axial modes. Next, we fine-tuned the mirrors to reduce the half axial mode's intensity and determined the finesse of the cavity $\mathcal{F}_{\text{exp}} = 75.91(15)$ by fitting Lorentz functions to a recorded spectrum. Assuming all resonator losses to stem from a mirror reflectivity of $98 \%$, we calculated an upper limit of $\mathcal{F}_{\text{theo}} = 155.50$. Clearly, other losses are present in the system, such as absorption on dust particles. Also, mirrors degrade with time, reducing their reflectivity.

Leaving the confocal condition, we observe higher mode excitations. We can correctly identify TEM$_{00}$, TEM$_{01}$ and TEM$_{10}$ modes taking an optical image with a CCD camera and fitting Hermite-Gaussian polynomials to projected intensities. We also took a picture of what could be a TEM$_{02}$ mode, however it is vertically not completely contained in the image, so a definitive identification is difficult. Two of the four pictures show saturation of the CCD intensity, which could be remedied by using a filter.

Lastly, we determined the mode spacing of higher modes by fitting Lorentz functions to a recorded spectrum. We measure it to be $\Delta\nu = 88.03(8) \unit{MHz}$. We can theoretically explain this result assuming the only excited TEM$_{mn}$ modes satisfy $m + n = 2$, which is highly unlikely. Another possibility is that we did not capture a full spectral range of the cavity, falsifying our conversion from time to frequency domain. 