% !TeX root = Bericht.tex
% !TeX spellcheck = en_US
\section{Introduction}
What would a world without laser technology look like? Well, it would look really different from the world we know. From GPS to accurate clocks, lasers are omnipresent in the modern world. But not only technical applications are worth mentioning here; they also play an indispensable role in various physical experiments, shedding light on phenomena ranging from quantum mechanics to material science. In this experiment, we focused on the laser beam's characteristics and its coupling to an optical cavity.

In this analysis, our focus is twofold. Firstly, we investigated the propagation behavior of the laser beam. We did this by measuring the beam's diameter at various distances and comparing with theoretical predictions. Secondly, we focused on the mode behavior of a cavity. To accomplish this, we had to align the laser's mode to the one of the cavity, using a lens. Finally, we deliberately misaligned the beam to get a view of higher order modes of the beam. A CCD camera was used optically capture these, whilst the spectrum was recorded on an oscilloscope via a photodiode.

Following a brief theoretical overview and an explanation of the applied theory, this report will present and discuss the evaluated data.
