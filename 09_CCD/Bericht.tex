% !TeX program = xelatex
\documentclass{alex_bericht}
\usepackage[export]{adjustbox}

\def\NFe{N_{\text{Fe}}}
\def\NH{N_{\text{H}}}

\begin{document}
%Seiten ohne Kopf- und Fußzeile sowie Seitenzahl
\pagenumbering{Roman}
\def\settitle{Eigenschaften optischer CCDs}
% !TeX root = Bericht.tex
% !TeX spellcheck = de_DE
\thispagestyle{empty}
\titlehead{\includegraphics[width=5cm]{logo.jpg}}
\title{Bestimmung der spezifischen Ladung von Elektronen}
\author{Mathias Gschnitzer\thanks{\href{mailto:mathias.gschnitzer@uibk.ac.at}{mathias.gschnitzer@student.uibk.ac.at}},
		Alexander Helbok\thanks{\href{alexander.helbok@student.uibk.ac.at}{alexander.helbok@student.uibk.ac.at}}, 
		Maximilian Märk\thanks{\href{maximillian.maerk@student.uibk.ac.at}{maximillian.maerk@student.uibk.ac.at}}}
\date{\today}
\maketitle
\vfill 

\section*{Abstract}
In diesem Versuch wird auf die Funktionsweise eines CCD-Sensors eingegangen, mite besonderem Augenmerk auf die Kalibrierung und Korrektur der Rohdaten. Dabei wird das Rohbild schrittweise mittels Bias-, Flatflied- und Dunkelstromaufnahmen kalibriert and analysiert. Dabei können wir den Gain-Faktor des CCD-Sensors, sowie das Ausleserauschen und die full well capacity bestimmen und mit dem Datenblatt vergleichen. Zuletzt bestimmen wir die Bandlücke von Silizium und des Donatorelements.
\vspace{1.5cm}


%Inhaltsverzeichnis
{\hypersetup{linkcolor=black}
\clearpage
\tableofcontents

%Verzeichnis aller Bilder
%\listoffigures

%Verzeichnis aller Tabellen
%\listoftables}
\cleardoubleoddpage

\pagenumbering{arabic}
% !TeX root = Bericht.tex
% !TeX spellcheck = de_DE
\section{Einleitung}
Halbleiterbauteile sind ein essentieller Bestandteil der heutigen Welt. Unmöglich wären moderne Computer und Mobiltelefone, sowie jede andere Art der Elektronik ohne den Einsatz von Halbleitern, ja selbst eine moderne Modelleisenbahn kommt ohne Sie nicht aus. Doch wo finden sich Halbleiter in elektrotechnischen Bauteilen? Prominente Vertreter sind verschiedenste Dioden, MOSFETs und Transistoren. Die zahlreichen Anwendungsfelder werfen die Frage nach der Besonderheit der Eigenschaften von Halbleitern auf. Einige Eigenschaften einer Halbleiter-Probe werden in diesem Versuch untersucht. \newline
In diesem Bericht wird nach einer kurzen Beschreibung der nötigen Theorie die Durchführung des Versuches erläutert. In der Datenauswertung werden mithilfe der „van der Pauw“-Methode Spannungen gemessen und mit Magnetfeldmessungen kombiniert .Der Hall-Koeffizient und der spezifische Widerstand einer Probe lassen sich daraus berechnen, um anschließend die Ionisationsenergie der Donatoren bzw. Akzeptoren zu bestimmen und den Ladungsträgerüberschuss in Abhängigkeit der inversen Temperatur aufzutragen. Aus diesem sowie dem temperaturabhängigen Verhalten lässt sich das Material der Probe bestimmen. Weiters wird noch ein anderer Aspekt in der Datenauswertung beleuchtet, und zwar wird der dominante Streumachanismus im Hoch- und Niedertemperaturbereich bestimmt. Zusammengefasst bieten bei diesem Versuch simple Messungen, nämlich Spannungsmessungen, einen spannenden Einblick in die Halbleiterphysik.


% !TeX root = Bericht.tex
% !TeX spellcheck = en_US
\section{Theory}
\label{sec:theorie}

In this section, the basic theoretical principles are briefly outlined to provide a better understanding of what a Gaussian beam is and how a cavity works.

\subsection{Gaussian beam}
The Gaussian modes are a set of solutions to the Helmholtz equation, which can be derived from Maxwell's equations using the paraxial approximation (see \autocite{gaussian_beam_script}). Their intensity is given by 
$$ I(x,y,z) \sim |U(x,y,z)|^2=I_0\left(\frac{w_0}{w(z)}\right)^2 \cdot \exp\left(-2\frac{x^2+y^2}{w(z)^2}\right)$$
, where $w_0$ is the beam radius when in focus (also known as the beam waist), $w(z)$ is the radius of the beam at a given location $z$, $I_0$ is the original intensity of the laser and $y$ is the spacial expansion. The beam width is given by
\begin{equation}\label{eqn:wz}
	w(z)=w_0\sqrt{1+\left(\frac{z}{z_R}\right)^2}\ .
\end{equation}
Here $z_R = \pi\cdot w_0/\lambda$ is called the Rayleigh length, which is given by the distance at which the intensity decreases by the factor $1/e^2$, at this point the beam width increases by $\sqrt{2}$. Via geometrical analysis, expressions for the curvature of the beam $R(z)$ and the divergence angle $\theta$ can be found (see \autocite{gaussian_beam_script}).

The Rayleigh length divides the beam into two regimes, a near field, where the beam can be regarded as a plane wave (curvature $R=\infty$) and far field, where the beam can be regarded as a cubic wave ($R(z)\sim z$). 

\subsection{Matrix formalism}\label{subsec:matrix}
In order to describe the effect of optical elements on a Gaussian beam, the ABCD matrix formalism can be used. Therefore, we define a complex quantity $q(z)$, such that
$$\frac{1}{q(z)} = \frac{1}{R(z)}- i\frac{\lambda}{\pi w(z)^2}.$$
When passing an optical element, the complex quantity changes according to
$$\frac{1}{q_2}=\frac{C+D/q_1}{A+B/q_1}.$$
The parameters $A,B,C,D$ are entries of a $2 \times 2$ matrix. The form of the matrices for free propagation, lenses and mirrors are given in \autoref{fig:matrix_optics_table}. Combination of optical elements is done via matrix multiplication, where the multiplication order corresponds to the order of propagation.

\begin{figure}[H]
	\centering
	\includegraphics[width=0.55 \textwidth]{Bild_2024-04-11_005434519}
	\caption{Matrix entries for lenses, mirrors and free propagation in space. Table taken from \autocite{gaussian_beam_script}.}
	\label{fig:matrix_optics_table}
\end{figure}

Using this formalism, we can calculate the beam waist $w_2$ of an incident collimated beam with waist $w_1$ and curvature $R = \infty$ at a distance $z = f$ after passing through a lens with focal length $f$. The result

\begin{equation}\label{eqn:w2}
	 w_2 = \sqrt{\frac{f^2w_1^2\lambda^2}{\pi^2w_1^4 + f^2\lambda^2}} \approx \frac{f\lambda}{\pi w_1}
\end{equation}
can be further simplified, assuming $f^2\lambda^2$ to be negligibly small.

\subsection{Optical Resonator}
\label{subsec:OptRes}
In an optical resonator, two mirrors are set up parallel to each other at a certain distance $L$. Due to the fact that the electric field has a node at each mirror, only waves with a wavelength $\lambda$ that is an integer inverse of the half distance between those mirrors can exist as standing waves. So we get the general relation
$$\lambda_n=\frac{2 L}{n}, $$ 
where $n$ is an integer. Changing wavelength to frequency $\nu$ and introducing the free spectral range (FSR) as $\Delta \nu_{\mathrm{FSR}}= c/2L$ the relation can be simplified to $\nu=n\Delta \nu_{\mathrm{FSR}}$ (see~\autocite{diodenlaser}).
If the mirrors are semi-transparent, monochromatic light can be transmitted. Depending on the reflectivity, other frequencies can also pass through the mirror and the peaks smooth out \autocite{Fabry-Perot_Interferometer_Tutorial}. 
%This behavior is shown in \autoref{fig:Reflexivität}. 

%\begin{figure}[H]
%	\centering
%	\includegraphics[clip,trim=0 0 0 2, width=0.9\linewidth]{Bild Reflektivität}
%	\caption{\glqq Mode spectrum of a Fabry-Pérot interferometer for mirror reflectances of 99.7\%, 80\%, and 4\%, illustrated by a blue, red, and green curve, respectively. 99.7\% reflectance corresponds to the case for the SA200 series, which has a free spectral range of 1.5 GHz. The 4\% reflectance corresponds to a typical "fringing effect" arising from reflections between parallel surfaces on glass plates.\grqq 
%		From \autocite{Fabry-Perot_Interferometer_Tutorial}.}
%	\label{fig:Reflexivität}
%\end{figure}

To quantify the quality of a cavity, we introduce the finesse $\mathcal{F}$ of a cavity, defined as the ratio between the FSR and full width at half minimum (FWHM) of the transmitted peaks. Assuming all losses coming from mirror transmission, we can theoretically compute the finesse. Both relations are listed below
\begin{equation}\label{eqn:finesse}
	\mathcal{F}_{\text{exp}} = \frac{\Delta \nu_{\mathrm{FSR}}}{\text{FWHM}} \qquad \mathcal{F}_{\text{theo}} = \frac{\pi\sqrt{R}}{1 - R},
\end{equation}
with $R$ being the mirror reflectivity.

A confocal resonator has a special geometry, where both mirrors are curved with the same radius $r$ as the length $L$ between them ($r = L$). This has some essential benefits, such as the fact that when correctly aligned the stability is very high. Additionally, if the laser beam is misaligned, we only get one peak between the two expected beams (see \autocite{WikiOticalCavit}) . 

A resonator decomposes an incident laser beam into its internal modes. To achieve maximum spatial overlap of laser and resonator mode, the laser mode has to be adjusted using lenses. To calculate the needed beam waist, we use
\begin{equation}\label{eqn:w0}
	2z_\mathrm{R} = \frac{2\pi w_0}{\lambda} = \sqrt{L(2r-L)}.
\end{equation}
Here, a symmetrical resonator is used with equal mirror curvatures $r$. 

If the laser and cavity modes do not match, apart from the fundamental transverse electro-magnetic mode (the TEM$_{00}$), also higher modes (TEM$_{mn}$ with $m, n > 0$) are excited. These excited states can be constructed by multiplying the ground state with Hermite-polynomials $H_i$, resulting in so called Hermite-Gaussian-beams:
\begin{equation}
	U_{\mathrm{(m,n)}} \sim {U_{(0,0)}H_{\mathrm{m}}\left(\frac{\sqrt{2}}{w(z) }x\right) H_{\mathrm{n}}\left(\frac{\sqrt{2}}{w(z)}y\right)}\label{hermite-gaussian-beams}
\end{equation}
Each revolution in the cavity gives the modes an additional phase shift, which changes the resonance condition from $\nu = n\Delta \nu_{\mathrm{FSR}}$ to
\begin{equation}\label{eqn:cos}
	\nu_{q,m,n} = \left[ (q+1)+\frac{1}{\pi} (m + n + 1)\arccos{\left(1-\frac{L}{r}\right)} \right]\Delta \nu_{\mathrm{FSR}}. 
\end{equation}
In case of a confocal resonator ($r = L$) the relation simplifies to
\begin{equation}
	\nu_{q,m,n} = \left[ (q+1)+\frac{1}{2}(m+n+1) \right]\Delta \nu_{\mathrm{FSR}}. 
\end{equation}
This  leads to half-axial-modes of a confocal resonator, as shown in the spectrum of a near planar, a confocal and a near concentric cavity in \autoref{fig:spektrum_cavities}. 


\begin{figure}[H]
	\centering
	\includegraphics[clip,trim=0 0 0 2, width=0.9\linewidth]{Bild_2024-04-11_001906466}
	\caption{\blockquote{Resonance frequencies of higher order modes in
			optical resonators for different configurations.
		}\cite{gaussian_beam_script} Figure taken from \cite{gaussian_beam_script}}
	\label{fig:spektrum_cavities}
\end{figure}




%
%% !TeX root = Bericht.tex
% !TeX spellcheck = de_DE
\section{Durchführung}

%\begin{figure}[H]
%    \centering
%    \includegraphics[width=\linewidth]{BildExberimentAufbauBeschriftung.JPG}
%    \caption{In dieser Abbildung sind der Kryostat, die Magnetspulen sowie die Sonde zur Messung des Magnetfeldes zu erkennen.}
%    \label{ExperimetAufbau}
%\end{figure}



Im Anschluss an die Installation des Zählrohrs wurde der Messaufbau mit dem Computer verbunden und das zugehörige Bedienungsprogramm gestartet. Der Aufbau ist in \autoref{aufbau_beschriftet} dargestellt. Zu erkennen ist auf der linken Seite der Draht, der durch Erhitzen durch Anlegen einer Anodenspannung und eines Stromflusses dazu gebracht wird, Elektronen zu emittieren. Durch die Blende kommen die Röntgenquanten in den rechten Raum des Geräts. Hier ist der schwenkbare Arm mit Winkelskala erkenntlich. Der Winkel wird relativ zur Horizontalen gemessen. Am Ende des Arms wird ein Geiger-Müller-Zählrohr befestigt, das mithilfe eines Kabels mit dem GM-Anschluss verbunden wird. Außerhalb des Geräts wird dieser Kabelanschluss mit dem Computer zur Datennahme verbunden. Am Gelenk des Schwenkarms befindet sich ein Steckplatz, auf dem ein Kristalle zur Aufspaltung der Röntgenquanten in ihre Wellenlängen angebracht werden kann. 
\begin{figure}[H]
	\centering
	\includegraphics[width=\linewidth]{aufbau_beschriftet.jpg}
	\caption{Versuchsaufbau: 1) Anode für Elektronenemission 2) Geiger-Müller-Zählrohr 3) Schwenkbarer Arm mit Winkeleinstellung 4) Kristall 5) Blende.}
	\label{aufbau_beschriftet}
\end{figure}




\subsection{ Zählrohrcharakteristik}
Zur Charakterisierung des Zählrohrs wurde zunächst eine manuelle Sondierung durchgeführt, um die ungefähren Positionen der einzelnen Bereiche zu bestimmen. Dazu wurde der Winkel des Zählrohrs auf \ang{5.5} eingestellt, um eine Sättigung bei maximaler Zählrohrspannung von $500 \unit{V}$ zu vermeiden. Als Beschleunigungsspannung im Strahlungsemitter wurde $35 \unit{kV}$ bei einem Emissionsstrom von $1 \unit{mA}$ gewählt. Die Zählrohrspannung wurde dabei auf $500 \unit{V} $ gesetzt. Nach der Aussonderung wurde eine Schrittweite von 10 zwischen $300 \mathrm{V}$ und $310 \mathrm{V}b$, von $1  \mathrm{V}$ zwischen $310  \mathrm{V}$ und $320 \mathrm{V}$, von $5 \mathrm{V}$ zwischen $320 \mathrm{V}$ und $350 \mathrm{V}$ und von $25 \mathrm{V}$ zwischen $350 \mathrm{V}$ und $500 \mathrm{V}$ gewählt. Der Grund hierfür ist, dass wir im am höchsten aufgelösten Bereich das charakteristische Ereignis untersuchen können und außerhalb dieses Bereichs ein konstanter Wert erwartet wird. Zu jedem Schritt der Zählrohrspannung werden drei angezeigte  Ereigniszahlen notiert. Da die Anzahl an Ereignissen auch bei konstanten Einstellungen zeitlich fluktuiert, werden für statistische Zwecke drei Werte pro Einstellung gemittelt. 

\subsection{Charakteristische Röntgenstrahlung}
Zunächst wurde der Kristall an der gewünschten Stelle eingebaut. Hierbei wurde ein LiF-Kristall (Lithiumfluorid) mit einer Gitterkonstante $d=201 \unit{pm}$ gewählt. Dieser Kristall wurde gewählt, da er die kleinste zur Verfügung stehende  Gitterkonstante aufwies, was eine feinere Auflösung der Wellenlängen ermöglichte. Die Winkelpositionierungsmöglichkeit des Versuchsaufbaus war auf den Wert \ang{0,1} beschränkt. Um eine automatisierte Messung vorzunehmen, wurde das Winkelverhältnis der Schwankung des Zählrohrs zur Schwankung des Kristalls auf 2 gesetzt. Das bedeutet, dass der eingestellte Winkel des Zählrohrs immer doppelt so groß ist wie der des Kristalls. Der Emissionsstrom wurde mit $I_\mathrm{Em}=1 \unit{mA}$ bei einer Beschleunigungsspannung von $U_\mathrm{Besch}=35 \unit{kV}$ bestimmt. Das Spektrum wurde zwischen \ang{5.5} und \ang {35.5} in Schritten von \ang{0.1} aufgetragen. Die Messung erfolgte an jedem Punkt für zwei Sekunden. Da auch Ereignisse berücksichtigt werden müssen, die nicht zuerst am Kristall gebeugt werden, wurde der gleiche Bereich unter den bekannten Parametern ebenfalls ohne den Kristall gemessen. 

\subsection{Bestimmung des Planckschen Wirkungsquantums}

Im Rahmen der Untersuchung war die kleinste emittierte Wellenlänge $\lambda_\mathrm{min}$ zu bestimmen. Zu diesem Zweck wurde zunächst der Anfangswinkel abgeschätzt, indem der Winkel manuell angepasst wurde, bis ein klarer Anstieg der Zählereignisse erkennbar war. Da bei steigender Spannung eine Verschiebung hin zu einer größeren Wellenlänge erwartet wird, wurde für jede Spannung extra sondiert. Die Messung wurde abgebrochen, sobald auf dem Bildschirm ein klarer Anstieg der Zählereignisse ersichtlich war. Es wurde darauf geachtet, dass eine hinreichende Anzahl an Messpunkten im Plateau vor dem Anstieg vorliegt, um eine Charakterisierung zu ermöglichen.  Die Messung wurde für Beschleunigungspannungen zwischen $15 \unit{V}$ und $30 \unit{V}$ in $2.5 \unit{V}$ Schritten durchgeführt. Der Emissionsstrom lag dabei bei $1 \unit{mA}$. 

\subsection{Absorptionsgesetz für Röntgenstrahlung}
Hierzu wurden Aluminiumfolien mit den Dicken $0.02 \unit{mm}$, $0.04 \unit{mm}$, $0.06,  \unit{mm}$, $0.08 \unit{mm}$ und $0.1 \unit{mm}$ und Zinkfolion mit den Dicken $0.025 \unit{mm}$, $0.05 \unit{mm}$, $0.75 \unit{mm}$, $0.1 \unit{mm}$ und $0.1 \unit{mm}$. 
%
\section{Ergebnisse}
Das durch den Ferromagneten verursachte Magnetfeld ohne äußeres Einwirken wurde zu Beginn der Messungen auf $1.045(1) \unit{mT}$ bestimmt. Stellt man ein Magnetfeld von $190 \unit{mT}$ an Computer ein, misst das Teslameter $252.6(1) \unit{mT}$. Die Unsicherheiten sind hier in der letzten angezeigten Stelle des Teslameters. Am Ende der Messungen wurde das ungetriebene Magnetfeld gemessen um einen möglichen Drift (durch Magnetisierung des Eisens) festzustellen und es wurde ein Wert von $0.979(1) \unit{mT}$ gemessen. Dieser Effekt ist also vernachlässigbar klein. Der angelegte Strom von $I = 500 \unit{\micro A}$ wird während des gesamten Versuches nicht verändert und wird mit einem Fehler von $1.05\% \pm 0.5 \unit{\micro A} $ bestückt, welcher im Handbuch nachgelesen werden kann. Das sind genau $8 \unit{\micro A}$ und ist unter (späterer) genauerer Betrachtung der Fehlerkontributionen der dominierende Faktor.

Die aufgezeichneten Widerstands- sowie Spannungswerte wurden mit einer Unsicherheit von $0.05 \unit{\ohm}$ und $0.05 \unit{mV}$ respektive gewählt. Das Multimeter hat zwar eine Anzeige mit drei Nachkommastellen, die Werte verändern sich aber sehr schnell, was eine genauere Bestimmung durch händisches Ablesen und Aufschreiben sehr schwierig macht.

Zu Beginn werden die gemessenen Widerstandswerte mittels einer Kalibrierungskurve \autocite{hall} in Temperaturen umgerechnet (unter Berücksichtigung der Fehlerpropagation). Wir kommen bei normaler Zimmertemperatur nach Konversion in Temperaturen auf $T = 324.81(4) \unit{K}$, was etwa $25 \unit{K}$ über dem erwarteten Wert für Raumtemperatur liegt (es wurde leider keine Temperaturmessung mit einem anderen Messgerät durchgeführt, um die Richtigkeit der Kalibrierungskurve zu validieren). Da wir aber bei allen Widerstandsmessungen physikalisch sinnvolle Temperaturen erhalten (nichtnegativ, monoton fallend) wird mit einer potentiell fehlerhaften Kalibrierungskurve fortgefahren.

Folgend werden Widerstands- und Temperaturmessung synonymisch verwendet. Da die Temperaturmessungen immer nach vier Spannungsmessungen durchgeführt wurden, wurde über die Temperatur interpoliert. Um Messungen für den spezifischen Widerstand und den Hall-Koeffizienten später zu kombinieren, wurden die Temperaturen hier wieder gemittelt.

Um genauere Spannungsmessungen zu erhalten wurden immer zwei Messungen mit unterschiedlicher Stromrichtung durchgeführt und anschließend gemittelt.

Der Hall Koeffizient wurde mittels \autoref{eq:RH} bestimmt. Hier ist zu beachten, die Berechnung der Spannungsdifferenz richtig auszuführen, da sich hier leicht ein Vorzeichenfehler einschleichen kann. Der Hall-Koeffizient $R_{\text{H}}$ liegt bei \blockquote{Raumtemperatur} ($T = 324.81(4) \unit{K}$) bei
$$R_{\text{H}} = -0.0398(6) \unit{\cubic\m\per\C} .$$

$R_{\text{H}}$ hängt gemäß \autoref{eq:RHn} über der Elementarladung $e$ direkt mit der Ladungsträgerdichte $n$ zusammen, wobei das Vorzeichen vom Hall Koeffizienten die Art/Polarität dieser festlegt. Das negative Vorzeichen lässt auf n-Dotierung schließen, was einen Elektronenüberschuss impliziert mit Ladungsträgerdichte
$$ n = -\frac{1}{R_\mathrm{H}e} = 1.57(3) \cdot 10^{20} \unit{\per\cubic\meter} .$$

Den spezifischen Widerstand kann man über \autoref{eq:roh} bestimmen. Hier ist das Vorzeichen der Spannungen unwichtig, da diese im Betrag betrachtet werden (negatives $\rho$ ist unphysikalisch). Wir kommen bei Raumtemperatur auf
$$ \rho = 0.0802(13) \unit{\ohm\meter} . $$

Aus $R_H$ und $\rho$ lässt sich die Beweglichkeit der Ladungsträger $\mu$ über \autoref{eq:mu} bestimmen.
$$ \mu = 4961(13) \unit{\square\cm\per\s\per\V}. $$

Das Verhalten der freien Ladungsträger kann theoretisch modelliert werden \autocite{hall} und man erwartet sich ein konstantes Verhalten bei Raumtemperatur und einen exponentiellen Abfall bei niedrigen Temperaturen, wobei ein kontinuierlicher Übergang zwischen den beiden stattfindet. Um das exponentielle Verhalten sichtbarer zu machen, wurde die Ladungsträgerdiche $n$ logarithmiert, wobei zu beachten ist, dass $n$ in einheitenlose Form gebracht werden muss. In \autoref{fig:plot1} ist $\ln(n \cdot \oldunit{m^3})$ auf die inverse Temperatur $T^{-1}$ aufgetragen.

\begin{figure}[H]
    \centering
    \includegraphics[width=\textwidth]{plot1.pdf}
    \caption{Der Logarithmus der bestimmten Ladungsträgerdichte $ n $ ist auf die inverse Temperatur $T^{-1}$ aufgetragen. Die Messpunkte sind mit $1\sigma$ Fehlerbalken in beiden Koordinaten ausgestattet. In orange wurde der gewichtete Mittelwert der ersten 6 Datenpunkten gebildet, wobei die dafür verwendeten Messwerte rot markiert sind. In blau wurde eine Gerade mittels ODR an die hellblau  gekennzeichneten Messwerte angepasst. Beide Geraden sind mit einem $2\sigma$ Konfidenzband bestückt.}
    \label{fig:plot1}
\end{figure}

Man erkennt das konstante Verhalten von $n$ bei Raumtemperatur (niedriger inverser Temperatur!) und das linear abfallende Verhalten vom Logarithmus sehr gut. Dazwischen befindet sich eine Übergangszone. Am unteren Ende der Temperatur weichen die Messwerte stark vom linearen Trend ab, da wir hier von dem PT100 Widerstand limitiert sind. Um den Ladungsträgerüberschuss zu messen, wurde aus Daten im konstanten Regime der Mittelwert gebildet. Dieser und die dafür verwendeten Messwerte sind in \autoref{fig:plot1} in orange zu sehen. Die Bandlücke $E_\mathrm{d}$ lässt sich mittels \autoref{eq:Ed} aus der Steigung der in blau angepassten Geraden berechnen. Die Geradenanpassung wurde über orthogonal distance regression (ODR) \autocite{odr} ausgeführt, um Fehler sowohl in x-, als auch in y-Richtung zu berücksichtigen. Wir erhalten folgende Werte für Ladungsträgerüberschuss und Bandlücke:
$$ N_\mathrm{D} - N_\mathrm{A} = 1.544(11) \cdot 10^{20} \unit{m^{-3}} \quad\text{und}\quad E_\mathrm{d} = 0.0091(4) \unit{eV} .$$

Aus $R_H$ und $\rho$ lässt nun sich die Beweglichkeit der Ladungsträger $\mu$ über \autoref{eq:mu} bestimmen. Diese ist in \autoref{fig:plot2} gegen die inverse Temperatur $T^{-1}$ geplottet.

\begin{figure}[H]
    \centering
    \includegraphics[width=\textwidth]{plot2.pdf}
    \caption{Die Beweglichkeit $\mu$ ist gegen die inverse Temperatur $T^{-1}$ aufgetragen. Die Messwerte sind mit $1\sigma$ Fehlerbalken bestückt, wobei diese bei hohen Temperaturen (und daher niedrigen inversen Temperaturwerten) sehr klein und in der Abbildung schlecht sichtbar sind.}
    \label{fig:plot2}
\end{figure}

Man erkennt in \autoref{fig:plot2}, dass $\mu$ mit abnehmender Temperatur ansteigt, ein Maximum bei etwa $110 \unit{K}$ erreicht und danach wieder absinkt. Aus der Theorie wissen wir (siehe \autoref{sec:Streuung}), dass verschiedene Mechanismen in einem Festkörper für die Veränderung der Beweglichkeit verantwortlich sind, die bei unterschiedlichen Temperaturen stattfinden. Diese hängen mit unterschiedlichen Potenzen von der Temperatur ab, weshalb ein doppellogarithmischer Plot der Beweglichkeit $\mu$ gegen die Temperatur $T$ in \autoref{fig:plot3} zu sehen ist.

\begin{figure}[H]
    \centering
    \includegraphics[width=\textwidth]{plot3.pdf}
    \caption{Der Logarithmus der Beweglichkeit $\mu$ ist auf den Logarithmus der Temperatur $T$ aufgetragen. In orange (blau) wurde eine Gerade an die Daten bei niedrigen (hohen) Temperaturen angepasst, die für die beiden Fits verwendeten Daten sind hier farblich markiert. Beide Geraden sind von $2\sigma$ Konfidenzbändern umgeben.}
    \label{fig:plot3}
\end{figure}

In \autoref{fig:plot3} sollten reine Potenzabhängigkeiten als Geraden mit der jeweiligen Potenz als Steigung erscheinen. Da zwei Streumechanismen erwartet werden, wurden zwei Geraden an die Daten angepasst, eine bei niedrigen Temperaturen (orange) und eine an Hohe (blau). Die Fits wurden wieder mit ODR durchgeführt und die für die Fitroutine verwendeten Daten wurden optisch passend gewählt und farblich markiert. Die Steigungen der beiden Geraden sind
$$ a_1 = 1.25(4) \quad\text{und}\quad a_2 = -1.130(8),$$

wobei $a_1$ die phononische und $a_2$ Rutherfordstreuung charakterisiert.

%
% !TeX root = Bericht.tex
% !TeX spellcheck = de_DE
\section{Schlussfolgerung}
In der Versuchsreihe konnte die Funktionsweise und Kalibrierung eines CCD-Sensors veranschaulicht werden. 

Der Gain-Faktor wurde auf zwei Arten bestimmt, einmal aus der statistischen Streuung bei Flatfield Aufnahmen, das zweite Mal exakter, durch einen Parabelfit. dabei erhalten wir für die grüne Folie Werte, die sich in der Unsicherheit entsprechen, bei der blauen Folie weichen die Werte voneinander ab. Die somit bestimmte maximale Elektronenanzahl ist mit der Herstellerangabe kompatibel. Gleiches gilt für das Ausleserauschen.
Der Blooming-Effekt, ein Sättigungseffekt bei CCD-Sensoren, konnte anschaulich dargestellt werden, dabei werden Elektronen in benachbarte Potentialtöpfe entlang der Ausleserichtung verschoben, wenn der eigentliche Pixel gesättigt ist. 
Im letzten Teil wurden zwei Bandlücken ermittelt, die erste davon, \SI{1.1(2)}{eV}, deckt sich im Rahmen der Unsicherheit mit dem Literaturwert \SI{1.12}{eV} bei \SI{300}{K} \autocite{bandluecke} für die Bandlücke von Silizium. Im Datenblatt wird die genaue Dotierung des CCD-Sensors nicht erwähnt, was das Zuordnen der zweiten Bandlückenenergie von $-0.34 \unit{eV}$ schwierig macht.
%
% !TeX root = Bericht.tex
% !TeX spellcheck = de_DE 
\section{Appendix}\label{sec:appendix}

\begin{figure}[H]
	\centering
	\includegraphics[width=\textwidth]{Ex4_3}
	\caption{Der Massenabsorptionskoeffizient \( \mu/\rho \) ist für Kupfer (oben) und Nickel (unten) auf die Energie sowie Wellenlänge aufgetragen. Ab einer bestimmten Energie \( E_{ion} \) ändert sich das Verhalten sprunghaft. Danach lässt sich eine \( 1/E^3 \) Abhängigkeit beobachten, was durch eine Fitfunktion in Rot angedeutet wird. Substrukturen wurden aus dem Fit ausgenommen.}
	\label{fig:plot3}
\end{figure}

\begin{figure}[H]
	\centering
	\includegraphics[width=\textwidth]{Ex4_2}
	\caption{Der Massenabsorptionskoeffizient \( \mu/\rho \) ist für Zinnober (oben) und Aluminium (unten) auf die Energie sowie Wellenlänge aufgetragen. Gegen höhere Energien lässt sich eine \( 1/E^3 \) Abhängigkeit beobachten, was durch eine Fitfunktion in Rot angedeutet wird. Substrukturen wurden aus dem Fit ausgenommen.}
	\label{fig:plot3}
\end{figure}

\begin{figure}[H]
	\centering
	\includegraphics[width=\textwidth]{Ex4_Zn}
	\caption{Der Massenabsorptionskoeffizient \( \mu/\rho \) ist für Zink auf die Energie sowie Wellenlänge aufgetragen.}
	\label{fig:plot3}
\end{figure}

%%Literaturverzeichnis
\printbibliography
\clearpage

% !TeX root = Bericht.tex
% !TeX spellcheck = de_DE
\section*{Erklärung}

Hiermit versichern wir, dass der vorliegende Bericht selbständig verfasst wurde und alle notwendigen Quellen und Referenzen angegeben sind.

\begin{tabular}{@{}p{2.5in}p{2.5in}@{}}
	\\[5\bigskipamount]
	& \hspace{2mm}\today \\[-12pt]
	\dotfill & \dotfill \\
	Alexander Helbok & Date \\[5\bigskipamount]
	& \hspace{2mm}\today \\[-12pt]
	\dotfill & \dotfill \\
	Jakob Hugo Höck & Date \\
	[5\bigskipamount]
	& \hspace{2mm}\today \\[-12pt]
	\dotfill & \dotfill \\
	Max Koppelstätter & Date \\
	\centering
\end{tabular}

\begin{minipage}{0.3\textwidth}
	\vspace{-20.5cm}
	\includegraphics[scale=0.3]{alex_sign}
\end{minipage}

\begin{minipage}{0.3\textwidth}
	\vspace{-13.4cm}
	\includegraphics[scale=1]{UnterschriftHugo}
\end{minipage}

\begin{minipage}{0.3\textwidth}
	\vspace{-7.4cm}
	\includegraphics[scale=0.2]{max_sign}
\end{minipage}

\end{document}
