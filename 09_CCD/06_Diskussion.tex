% !TeX root = Bericht.tex
% !TeX spellcheck = de_DE
\section{Schlussfolgerung}
In der Versuchsreihe konnte die Funktionsweise und Kalibrierung eines CCD-Sensors veranschaulicht werden. 

Der Gain-Faktor wurde auf zwei Arten bestimmt, einmal aus der statistischen Streuung bei Flatfield Aufnahmen, das zweite Mal exakter, durch einen Parabelfit. dabei erhalten wir für die grüne Folie Werte, die sich in der Unsicherheit entsprechen, bei der blauen Folie weichen die Werte voneinander ab. Die somit bestimmte maximale Elektronenanzahl ist mit der Herstellerangabe kompatibel. Gleiches gilt für das Ausleserauschen.
Der Blooming-Effekt, ein Sättigungseffekt bei CCD-Sensoren, konnte anschaulich dargestellt werden, dabei werden Elektronen in benachbarte Potentialtöpfe entlang der Ausleserichtung verschoben, wenn der eigentliche Pixel gesättigt ist. 
Im letzten Teil wurden zwei Bandlücken ermittelt, die erste davon, \SI{1.1(2)}{eV}, deckt sich im Rahmen der Unsicherheit mit dem Literaturwert \SI{1.12}{eV} bei \SI{300}{K} \autocite{bandluecke} für die Bandlücke von Silizium. Im Datenblatt wird die genaue Dotierung des CCD-Sensors nicht erwähnt, was das Zuordnen der zweiten Bandlückenenergie von $-0.34 \unit{eV}$ schwierig macht.