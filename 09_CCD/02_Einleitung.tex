% !TeX root = Bericht.tex
% !TeX spellcheck = de_DE
\section{Einleitung}
Wo wären wir ohne CCDs? Diese Frage wirkt zwar nicht alltäglich, sollte sie jedoch sein. Auch wenn dem einfachen Endnutzer einer Digitalkamera deren Funktion als peripher erscheint, ist deren Auswirkung auf den Erkentnisgewinn der letzten Dekaden unabschätzbar. Auch wenn die Familie der Halbleiterbauteile die Welt nachhaltig verändert hat, revolutionierte das CCD speziell die Bildgebung in der Astrophysik. 

Genug der schönen Worte, was wurde in diesem Versuch behandelt. Es wurden grundlegende Eigenschaften eines CCDs untersucht, welche zur Kalibrierung benötigt werden, um schlussendlich damit valide Messungen durchzuführen. Weiters wurden phänomenologische Eigenschaften betrachtet, wie die Temperaturabhängigkeit und das Blooming. Nach einem kurzen Theorie-Teil werden wir auf unsere Ergebnisse eingehen.